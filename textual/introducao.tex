\section{Introdução}
Em linhas gerais Democracia é um modelo politico originado na Grécia antiga, 
especificamente na cidade estado de Atenas e que tem suas principais bases 
definidas como "governo do povo" ou "governo da maioria". Um dos principais aspectos
da democracia é a soberania politica do povo, ou seja o povo elege seus representantes
através da instituição de voto. \par
Atualmente a grande maioria dos paises democráticos de direito adotam uma 
democracia representativa que é dividida em três áreas de poder: o Executivo, responsável por toda a organização
e direção do governo, o Legislativo, responsável por toda a legislação e o Judiciário,
responsável por todas as decisões judiciais. \par
Para garantir a eleição democrática de seus representantes é adotado o sufrágio 
universal, que consiste no pleno direito ao voto de todos os cidadãos adultos 
independentemente de nível de alfabetização, classe, renda, etnia ou sexo. \par
Muito além das criticas que podem ser realizadas a esta forma de governo, é válido
destacar que a democracia pressupõe como causula pétri a garantia ao cidadão de
que sua vóz será ouvida por aqueles que exercem o poder político em seu nome, 
com base nisto é inegavel que o processo eleitoral adquire importancia basilar
neste mesmo processo de representação. \par
Para a realização de um processo eleitoral justo e democrático, é elaborado um 
"sistema de votação", também chamado de "sistema eleitoral", que consiste em um
meio de escolha de um representante entre um certo numero de candidatos.
Um sistema eleitoral especifica a forma de cédula ou seja o meio em que o individuo
expressará seu desejo de voto, o conjunto de votos permitidos, ou seja o conjunto 
de candidatos, um método de mensuração e um algoritmo para determinar o 
resultado final. \par
Diversos países pelo mundo adotam diferentes sistemas eleitorais, como por exemplo
o sistema eleitoral brasileiro que desde 1996 utiliza sistemas eletrônicos de votação,
ou o sistema eleitoral de Portugal que utiliza cartões de voto impresso.
Em quase todos os sistemas eleitorais existentes, há grandes questionamentos sobre
a segurança do método adotado, em sistemas de votos com cédulas físicas por exemplo
como nem todos os eleitores possuem alfabetização muitas das vezes os votos são
preenchidos de maneira incorreta inviesando os resultados e confundindo os fiscais, 
sem destacar ainda o impacto ambiental causado pelo voto impresso já que exige
um gasto elevado com papel.\par
Por outro lado em países com sistemas eleitorais baseados em sistemas eletrônicos
como o Brasil existem questionamentos frequentes sobre a segurança do mecanismo 
ou a possíbilidade de fraudes já que em princípios técnicos como é elaborado
pelo próprio orgão regulamentador eleitoral o código poderia ser escrito a 
beneficiar o candidato A ou B gerando incerteza ao eleitor se seu voto
foi computado corretamente ao candidato em que votou. \par
Com base neste contexto pesquisar o uso de inovações técnológicas para garantir a 
confiabilidade nos processos de escolhas de representantes legais é de grande relevância
para a sociedade. A perca de confiabilidade nos sistemas eleitorais têm como
causas e consequências, vitórias de candidatos populistas, aumento considerável
de radicalismos em discursos (populares e políticos), percas e retrocessos dos
direitos sociais e até mesmo escândalos de corrupção institucionalizada. \par
Uma tecnologia que vem sendo muito estudada por conta de sua capacidade de 
descentralização e seu alto grau de confiabilidade é o blockchain, esta tecnologia
idealizada por Satoshi Nakamoto \cite[Bitcoin P2P e-cash paper]{nakamoto2008re}
rendeu valores exorbitantes à criptomoedas sem a necessidade de nenhuma organização
controladora ou um terceiro para garantir ou verificar as transações já que tudo
é determinado por um conjunto de regras estritamente matemáticas. \par
A problemática então levantada é a seguinte: A técnologia blockchain poderia contribuir
para formação de um sistema de voto democrático, uma espécie de criptocracia? \par
Afim de hipótese com base nas caracteristicas básicas dos sistemas blockchain, 
a desnecessáriedade de orgãos controladores, a capacidade de sigilo criptográfico
e transparência conferida ao meio, a tecnologia blockchain pode ser
utilizada para incrementar os processos de voto democrático ao redor do mundo 
substituindo tanto cédulas físicas como os atuais sistemas de voto eletrônicos 
não baseados em blockchain. Permitindo sistemas online de votação que teriam 
além da internet, os dispositivos dos próprios eleitores (Computadores, Smartphones, 
Tablets, etc) tornando desnecessária as votações  \textit{in loco}, o que agregaria 
economias absurdas de dinheiro público, aumentando a iniciativa popular e participação dos eleitores, tornando a democracia tão 
majoritariamente direta quanto possível. \par
\subsection{Objetivos}
\subsubsection{Objetivo geral}
Este presente documento tem como objetivo estabelecer uma relação evolutiva 
técnológica entre a sistemas eleitorais de voto democrático e tecnologias blockchain
para delinear uma cripto-democracia.
\subsubsection{Objetivos específicos}
\begin{itemize}
  \item Conceituar os principais aspectos e caracteristicas dos sistemas eleitorais democráticos
  existentes e o que torna um sistema eleitoral acessível, seguro e válido democraticamente.
  \item Compreender estruturas e caracteristicas dos sistemas blockchain e comparar com os sistemas
  eleitorais existentes.
  \item Correlacionar os sistemas eleitorais democráticos com as tecnologias blockchain.
  \item Identificar as alavancas de evolução disponíveis aos sistemas de voto democrático 
  com o uso de tecnologias blockchain.
  \item Estabelecer um sistema de voto democrático baseado em blockchain que atenda
  aos requisitos de segurança e confiabilidade coerentes com os princípios 
  democráticos representativos.
\end{itemize}
\clearpage
